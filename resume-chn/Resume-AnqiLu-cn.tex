%%%%%%%%%%%%%%%%%%%%%%%%%%%%%%%%%%%%%%%%%
% Friggeri Resume/CV
% XeLaTeX Template
% Version 1.1 (9/2/15)
%
% This template has been downloaded from:
% http://www.LaTeXTemplates.com
%
% Original author:
% Adrien Friggeri (adrien@friggeri.net)
% https://github.com/afriggeri/CV
%
% License:
% CC BY-NC-SA 3.0 (http://creativecommons.org/licenses/by-nc-sa/3.0/)
%
% Important notes:
% This template needs to be compiled with XeLaTeX and the bibliography, if used,
% needs to be compiled with biber rather than bibtex.
%
%%%%%%%%%%%%%%%%%%%%%%%%%%%%%%%%%%%%%%%%%

\documentclass[]{friggeri-cv-cn} % Add 'print' as an option into the square bracket to remove colors from this template for printing


\addbibresource{bibliography.bib} % Specify the bibliography file to include publications

\begin{document}

\header{卢}{安琪}{软件工程师,IBM中国研发中心} % Your name and current job title/field

%----------------------------------------------------------------------------------------
%	SIDEBAR SECTION
%----------------------------------------------------------------------------------------

\begin{aside} % In the aside, each new line forces a line break
\section{联系方式}
上海市
~
+86 138 1641 2974
~
\href{mailto:marine.luanqi@gmail.com}{marine.luanqi@gmail.com}
~
\href{https://github.com/anqilu}{github.com/anqilu}
\href{https://www.linkedin.com/in/anqilu}{linkedin.com/in/anqilu}
\section{语言}
英语 IELTS 7.5
西班牙语
\section{编程语言}
{\color{red} $\varheartsuit$} Python
GO, JAVA
JavaScript, C\#
\section{数据库}
MySQL
MongoDB
\section{开源技术}
OpenStack
Hadoop
SPARQL
\section{集成开发环境}
PyCharm
Eclipse
\section{编辑器}
Vim
\end{aside}

%----------------------------------------------------------------------------------------
%	WORK EXPERIENCE SECTION
%----------------------------------------------------------------------------------------

\section{工作经验}

\begin{entrylist}
%------------------------------------------------
	\entry
	{2015}
	{IBM国际商业机器(上海)投资有限公司}
	{上海,中国}
	{CDL软件开发
		\begin{itemize}
			\item 开发并维护PPIM (Pure Power Integrated Manager)。负责VirtualMachine管理模块。{\footnotesize\addfontfeature{Color=lightgray} Dojo, REST}
			\item 参加PPIM的POC项目Heimdallr。负责系统监控仪表盘模块。{\footnotesize\addfontfeature{Color=lightgray} RRD Tools}
			\item 参与RDO (OpenStack on Redhat, CentOS, Fedora)在PPC64架构上的解决方案。主要负责Neutron组件的移植,以及自动化部署OpenStack。{\footnotesize\addfontfeature{Color=lightgray} OpenStack, IBM Power, Virtualization}
		\end{itemize}
	}
%------------------------------------------------
\\
%------------------------------------------------
	\entry
	{2014}
	{EMC\textsuperscript{2}易安信信息技术研发(上海)有限公司}
	{上海,中国}
	{实验室软件工程师,实习生
		\begin{itemize}
			\item 开发与维护LMS2 (实验室管理系统II)。主要参与模块为网络管理模块,权限控制模块,仪表盘模块,用户ticket模块。 {\footnotesize\addfontfeature{Color=lightgray} CakePHP, MySQL, Apache}
			\item 开发网络应用程序,支持实验室用户的特定需求。{\footnotesize\addfontfeature{Color=lightgray} Python}
			\item 支持与维护实验室内的服务器以及存储设备软硬件。
		\end{itemize}
	}
%------------------------------------------------
\\
%------------------------------------------------
	\entry
	{2013}
	{上海众言网络科技有限公司}
	{上海,中国}
	{项目技术支持工程师,实习生
		\begin{itemize}
			\item 根据客户需求,开发定制的调研系统。{\footnotesize\addfontfeature{Color=lightgray} Django}
			\item 分析与处理调研数据。{\footnotesize\addfontfeature{Color=lightgray} NumPy, PANDAS}
			\item 管理调研数据。{\footnotesize\addfontfeature{Color=lightgray} MongoDB}
		\end{itemize}
	}
%------------------------------------------------
\end{entrylist}

%----------------------------------------------------------------------------------------
%	EDUCATION SECTION
%----------------------------------------------------------------------------------------

\section{教育背景}

\begin{entrylist}
%------------------------------------------------
	\entry
	{2014--2015}
	{硕士学位 {\normalfont ,计算机科学}}
	{爱丁堡大学,英国}
	{
		{\color{red} $\vardiamondsuit $}GPA: 3.75/4.0\\
		{\color{red} $\vardiamondsuit $}毕业项目:dispel4py性能测评机制。该项目为科学工作流系统dispel4py建立性能监控框架以及性能数据库,监控并分析时间,内存,数据流特征,工作流拓扑等参数,以优化工作流部署决策。\\
		{\color{red} $\vardiamondsuit $}研究论文:云计算中基于遗传算法的任务分配机制。该论文介绍基于优化的遗传算法的任务分配算法,并分析它们如何提升性能(缩短时间,降低能耗,平衡系统资源分配,提升系统资源利用率)。\\
		{\color{red} $\vardiamondsuit $}主要课程:
		高级数据库;并行编程语言及系统;分布式系统;极限计算;信息论;语义型网络系统;软件开发架构、流程及管理
	}
%------------------------------------------------
\\
%------------------------------------------------
	\entry
	{2010--2014}
	{学士学位 {\normalfont ,电子商务}}
	{上海财经大学,中国}
	{
		{\color{red} $\vardiamondsuit $}GPA: 3.38/4.0\\
		{\color{red} $\vardiamondsuit $}主要课程:
		程序设计基础;面向对象的程序设计;数据库;计算机网络;数据结构;管理信息系统;综合设计实验;高等数学;概率论与数理统计;运筹学;离散数学;高等代数
	}
%------------------------------------------------
\end{entrylist}

%%----------------------------------------------------------------------------------------
%%	PUBLICATIONS SECTION
%%----------------------------------------------------------------------------------------
%
\section{发表专利}

\printbibsection{patent} % Print all patents from the bibliography

%----------------------------------------------------------------------------------------
\end{document}
